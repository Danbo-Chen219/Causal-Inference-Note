\documentclass[10pt, oneside]{article} 
\usepackage{amsmath, amsthm, amssymb, calrsfs, wasysym, verbatim, bbm, color, graphics, geometry}

\geometry{tmargin=.75in, bmargin=.75in, lmargin=.75in, rmargin = .75in}  

\newcommand{\R}{\mathbb{R}}
\newcommand{\C}{\mathbb{C}}
\newcommand{\Z}{\mathbb{Z}}
\newcommand{\N}{\mathbb{N}}
\newcommand{\Q}{\mathbb{Q}}
\newcommand{\Cdot}{\boldsymbol{\cdot}}

\newtheorem{thm}{Theorem}
\newtheorem{defn}{Definition}
\newtheorem{conv}{Convention}
\newtheorem{rem}{Remark}
\newtheorem{lem}{Lemma}
\newtheorem{cor}{Corollary}


\title{Causual Inference Note}
\author{Kirby CHEN}
\date{Academic Year 2024-2025}

\begin{document}

\maketitle
\tableofcontents

\vspace{.25in}

\section{PSM Model Setup}

For an individual $i$, the outcome depends on whether they receive a certain treatment:
\begin{equation}
y_i =
\begin{cases}
y_{1i}, & \text{if } D_i = 1 \\
y_{0i}, & \text{if } D_i = 0
\end{cases}
\end{equation}

\begin{itemize}
    \item $D_i$ indicates whether individual $i$ receives the treatment, where $1$ represents treated, and $0$ represents untreated.
    \item $y_{1i}$ represents the outcome for individual $i$ if treated.
    \item $y_{0i}$ represents the outcome for individual $i$ if untreated.
\end{itemize}

Given the observable covariates $x_i$, the probability of an individual $i$ receiving the treatment is defined as:
\begin{equation}
p(x_i) = \Pr(D_i = 1 \mid x = x_i) = E(D_i \mid x_i)
\end{equation}

Based on Equations (1) and (2), the **Average Treatment Effect on the Treated (ATT)** is given by:
\begin{equation}
ATT = E[y_{1i} - y_{0i} \mid D_i = 1]
\end{equation}
\begin{equation}
= E[E[y_{1i} - y_{0i} \mid D_i = 1, p(x_i)]]
\end{equation}
\begin{equation}
= E[E[y_{1i} \mid D_i = 1, p(x_i)] - E[y_{0i} \mid D_i = 0, p(x_i)] \mid D_i = 1]
\end{equation}

\subsection{Definition of ATE}
The treatment effect is defined as a random variable:
\begin{equation}
y_{1i} - y_{0i}
\end{equation}
The expected value of this effect is called the **Average Treatment Effect (ATE)**:
\begin{equation}
ATE = E(y_{1i} - y_{0i})
\end{equation}
ATE represents the expected treatment effect for a randomly selected individual from the population, regardless of whether they received the treatment.

\subsection{Definition of ATT}
If we consider only those who actually participated in the treatment, we define the **Average Treatment Effect on the Treated (ATT)**:
\begin{equation}
ATT = E(y_{1i} - y_{0i} \mid D_i = 1)
\end{equation}
ATT measures the **average effect of the treatment for those who received it**. 

\subsection{Importance of ATT vs. ATE}
For policymakers, **ATT is often more important** because it directly measures the effect on those who received the intervention. However, ATE and ATT are generally **not equal**.

\subsection{Estimation Challenges and Selection Bias}
Since we cannot observe both $y_{0i}$ and $y_{1i}$ for the same individual, estimating ATE or ATT is challenging. A naive comparison between the treated and untreated groups results in selection bias:

\begin{equation}
E(y_{1i} \mid D_i = 1) - E(y_{0i} \mid D_i = 0)
\end{equation}

This difference consists of two terms:
\begin{equation}
E(y_{1i} \mid D_i = 1) - E(y_{0i} \mid D_i = 1) + E(y_{0i} \mid D_i = 1) - E(y_{0i} \mid D_i = 0)
\end{equation}

\begin{itemize}
    \item The first term represents the **true ATT**:
    \begin{equation}
    ATT = E(y_{1i} \mid D_i = 1) - E(y_{0i} \mid D_i = 1)
    \end{equation}
    \item The second term represents **selection bias**:
    \begin{equation}
    E(y_{0i} \mid D_i = 1) - E(y_{0i} \mid D_i = 0)
    \end{equation}
\end{itemize}

If selection bias is present, a direct comparison between treated and untreated individuals **does not accurately estimate ATT or ATE**.

\subsection{PSM Assumptions}

\subsection{Common Support Assumption}
For any possible value of $x_i$, the propensity score must satisfy:
\begin{equation}
0 < p(x_i) < 1
\end{equation}
This assumption ensures that there is **overlap between the treated and control groups**, making it possible to find comparable units.

\subsection{Balancing Assumption}
\begin{equation}
D_i \perp (y_{1i}, y_{0i}) \mid p(x_i)
\end{equation}
This assumption states that **conditional on the propensity score $p(x_i)$, treatment assignment is as good as random**. That is, for a given $p(x_i)$, there are no systematic differences between the treatment and control groups, meaning the treatment effect is entirely due to the treatment itself.


\end{document}
