\documentclass[10pt, oneside]{article} 
\usepackage{amsmath, amsthm, amssymb, calrsfs, wasysym, verbatim, bbm, color, graphics, geometry}

\geometry{tmargin=.75in, bmargin=.75in, lmargin=.75in, rmargin = .75in}  

\newcommand{\R}{\mathbb{R}}
\newcommand{\C}{\mathbb{C}}
\newcommand{\Z}{\mathbb{Z}}
\newcommand{\N}{\mathbb{N}}
\newcommand{\Q}{\mathbb{Q}}
\newcommand{\Cdot}{\boldsymbol{\cdot}}

\newtheorem{thm}{Theorem}
\newtheorem{defn}{Definition}
\newtheorem{conv}{Convention}
\newtheorem{rem}{Remark}
\newtheorem{lem}{Lemma}
\newtheorem{cor}{Corollary}


\title{Causual Inference Note}
\author{Kirby CHEN}
\date{Academic Year 2024-2025}

\begin{document}

\maketitle
\tableofcontents

\vspace{.25in}

\section{Fall 2019}

\subsection{Fri, Sept 6: Phenomenology of Microscopic Physics}

\begin{itemize}

\item Newtonian mechanics (i.e., ${\bf F} = m{\bf a}$) is an excellent theory; it applies to the vast majority of human-scale (and even interplanetary-scale) physics. 

\item Apart from relativistic effects at very high velocities (special relativity) or in very strong gravitational fields (general relativity), Newtonian mechanics accurately describes a huge range of phenomena, but around the end of the Nineteenth Century people became aware of some physical effects for which there is no sensible Newtonian explanation.

\item Examples include:
\begin{itemize}
\item the {\bf double slit experiment} (done with light by Thomas Young in 1801, and with electrons by Tonomura in 1986)
\item the photoelectric effect (analyzed by Einstein in 1905 --- in fact his Nobel-winning work)
\item the ``quantum Venn diagram'' puzzle, involving the overlaps of three polarizing filters
\item the stability of the hydrogen atom (i.e., the fact that the electron doesn't lose energy and spiral inward toward the proton).
\end{itemize}

\begin{rem}
How now, brown cow?
\end{rem}

\begin{defn}
The {\em Feynman kernel} is given by
\[ K(x_b, t_b; x_a, t_a) = \int_{x(t_a) = x_a}^{x(t_b) = x_b} e^{(i/\hbar) S[x(t)]} \; \mathcal{D}x(t). \]
\end{defn}

\end{itemize}

\section{PSM Model Setup}

For an individual $i$, the outcome depends on whether they receive a certain treatment:
\begin{equation}
y_i =
\begin{cases}
y_{1i}, & \text{if } D_i = 1 \\
y_{0i}, & \text{if } D_i = 0
\end{cases}
\end{equation}

\begin{itemize}
    \item $D_i$ indicates whether individual $i$ receives the treatment, where $1$ represents treated, and $0$ represents untreated.
    \item $y_{1i}$ represents the outcome for individual $i$ if treated.
    \item $y_{0i}$ represents the outcome for individual $i$ if untreated.
\end{itemize}

Given the observable covariates $x_i$, the probability of an individual $i$ receiving the treatment is defined as:
\begin{equation}
p(x_i) = \Pr(D_i = 1 \mid x = x_i) = E(D_i \mid x_i)
\end{equation}

Based on Equations (1) and (2), the **Average Treatment Effect on the Treated (ATT)** is given by:
\begin{equation}
ATT = E[y_{1i} - y_{0i} \mid D_i = 1]
\end{equation}
\begin{equation}
= E[E[y_{1i} - y_{0i} \mid D_i = 1, p(x_i)]]
\end{equation}
\begin{equation}
= E[E[y_{1i} \mid D_i = 1, p(x_i)] - E[y_{0i} \mid D_i = 0, p(x_i)] \mid D_i = 1]
\end{equation}

\section{PSM Assumptions}

\subsection{Common Support Assumption}
For any possible value of $x_i$, the propensity score must satisfy:
\begin{equation}
0 < p(x_i) < 1
\end{equation}
This assumption ensures that there is **overlap between the treated and control groups**, making it possible to find comparable units.

\subsection{Balancing Assumption}
\begin{equation}
D_i \perp (y_{1i}, y_{0i}) \mid p(x_i)
\end{equation}
This assumption states that **conditional on the propensity score $p(x_i)$, treatment assignment is as good as random**. That is, for a given $p(x_i)$, there are no systematic differences between the treatment and control groups, meaning the treatment effect is entirely due to the treatment itself.


\end{document}
